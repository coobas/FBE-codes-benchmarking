%!TEX root = fbe_codes_benchmarking.tex

\section{Introduction} % (fold)
\label{sec:intro}

As new large tokamak devices, such as ITER or DEMO, are being designed and constructed, free-boundary equilibrium codes become increasingly important. These codes are capable of simulating self-consistently the evolution of the plasma equilibrium and the poloidal field components of a tokamak. In particular, simulated can be the currents in active poloidal field (PF) coils and in all passive PF elements, such as the tokamak vessel. Taken into account are PF power supplies and the mutual inductive coupling between all PF elements and the plasma. This is in contrast to fixed-boundary equilibrium solvers, which do not take the PF systems into account. Results of free-boundary equilibrium simulations are therefore key for designing tokamak engineering parameters and operation scenarios. For example, the ${\mathbf{j}} \times {\mathbf{B}}$ forces set important requirements on the construction and, vice versa, design limits must be taken into account in the operation parameters. Operation scenarios must be designed to be compatible with the PF systems properties, such as the current and voltage limits of PF power supplies and coils or the Volt-second capacity of the central solenoid.

Besides stand-alone operation, coupling of equilibrium codes to transport solvers is very important. The ordering in MHD (magnetohydrodynamic) equations (see e.g. \cite{Jardin2011}) results in the time independent Grad-Shafranov equation, which describes the equilibrium (force balance), and the resistive current diffusion. Flux-averaging of MHD equations yields, a coupled system of the Grad-Shafranov equation---an elliptic partial differential equation for the poloidal magnetic flux $\psi\left( R,Z \right)$---and transport (advection-diffusion) equations for $\psi\left( t,\rho \right)$ and $p\left( t,\rho \right)$. (Alternatively, the current diffusion equation can be formulated for the rotational transform $\iota$.) Here, $R,Z$ refer to Cartesian coordinates in the poloidal plane, $t$ is time and $\rho$ is a flux surface coordinate based on the toroidal magnetic flux. The flux surface geometry is described by the solution of the Grad-Shafranov equation. 

Several free-boundary equilibrium (FBE) codes are currently available: TSC \cite{TSCref}, DINA \cite{DINA1993}, CREATE-NL \cite{CREATEref}, CORSICA \cite{CORSICAref}, SPIDER \cite{SPIDERref}, CEDRES++ \cite{CEDRESref}, FREEBIE \cite{FREEBIERef}. These codes are of course internally different but all of them should be capable of simulating the temporal evolution of tokamak equilibrium consistently with the evolution of the PF systems. Very importantly, FBE codes have never been rigorously benchmarked against each other. We therefore propose a set of well defined benchmark tasks for benchmarking free-boundary equilibrium codes. These tasks are detailed in Section~\ref{sec:description}. The benchmarks range from vacuum cases (i.e. without any plasma) over several static cases to dynamical evolution with and without feed-back control. Results for \emph{??? FREEBIE, CEDRES++, DINA, SPIDER, ???} are presented in Section~\ref{sec:results}. In this paper, we address exclusively stand-alone operation, i.e., the tasks that require coupling to transport equation are not carried out in this paper but will be addressed and published in a near future.

% section intro (end)


