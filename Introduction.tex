%!TEX root = fbe_codes_benchmarking.tex

\section{Introduction} % (fold)
\label{sec:intro}

As new large tokamak devices, such as ITER or DEMO, are being designed and constructed, free-boundary equilibrium codes become increasingly important. These codes are capable of simulating self-consistently the evolution of the plasma equilibrium and the poloidal field components of a tokamak. In particular, simulated can be the currents in active poloidal field (PF) coils and in all passive PF elements, such as the tokamak vessel. Taken into account are PF power supplies and the mutual inductive coupling between all PF elements and the plasma. This is in contrast to fixed-boundary equilibrium solvers, which do not take the PF systems into account. Results of free-boundary equilibrium simulations are therefore key for designing tokamak engineering parameters and operation scenarios. For example, the ${\mathbf{j}} \times {\mathbf{B}}$ forces set important requirements on the construction and, vice versa, design limits must be taken into account in the operation parameters. Operation scenarios must be designed to be compatible with the PF systems properties, such as the current and voltage limits of PF power supplies and coils or the Volt-second capacity of the central solenoid.

Several free-boundary codes are currently available: TSC, DINA, CREATE-NL, CORSICA, SPIDER, CEDRES++, FREEBIE \cite{FREEBIERef}. These codes are of course internally different but all of them should be capable of simulating the temporal evolution of tokamak equilibrium consistently with the evolution of the PF systems.

% section intro (end)


